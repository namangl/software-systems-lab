\documentclass{article}
\usepackage{hyperref}
\usepackage[margin=1.25in]{geometry}
% \usepackage{fancyhdr}
% \pagestyle{fancy}

\title{CSP302: Project Idea}
\author{Naman Goyal (2015CSB1021) \\
Sarthak Gupta (2015CSB1029) \\
Vishal Singh (2015CSB1040) \\
Nittin Singh (2015CSB1067)
}
\begin{document}

\begin{titlepage}
\begin{center}
\vspace{1cm}
\Large{CSP203 - Software Systems Lab}
\vfill

\line(1,0){400}\\[1mm]
\huge{\textbf{Project Proposals}}\\
\line(1,0){400}\\
\vfill
\Large{Naman Goyal (2015CSB1021) \\
Sarthak Gupta (2015CSB1029) \\
Vishal Singh (2015CSB1040) \\
Nittin Singh (2015CSB1067)
}\\

\vfill
\today
\end{center}
\end{titlepage}

\tableofcontents
\clearpage

\section{Commuter's Companion}

\paragraph{Objective}
To keep log of all travel related information at a central place
\paragraph{Target Audience}
Group of people travelling together
\paragraph{Features}
The app keeps the following records
\begin{itemize}
    \item Reservations in train/ buses and others relevant booking information
    \item All the travel related expenses
    \item Integrates with Google Maps to suggest nearby Bus Stops/ Eating Places/ Tourist Spots
\end{itemize}
\paragraph{Inspiration}
Scholars travelling from Ropar are currently unware about other fellow travelling via same mode. It would be boon if the app is able to read the SMS of Train reservation and suggest who is travelling with us.

\vspace{1cm}

\section{IIT Ropar App - Easy and Usable}
\paragraph{Objective}
To ease students' stay at IIT Ropar and provide one-click access to all relevant updates at the institute.
\paragraph{Target Audience}
Students of IIT Ropar
\paragraph{Features}
\begin{itemize}
    \item Know about ongoing events, conferences, talks, seminars - Cultural, Academic, Sports etc
    \begin{itemize}
        \item Add a new events
        \item Moderate Existing events
        \item Add Reviews
        \item Suggest events based on your interest
    \end{itemize}
    \item Exchange books, lab coats and other stuffs in organised model
    \item Exchange old Question Papers and serves as Question bank
    \item FAQs section about
    \begin{itemize}
        \item The ``hacks" at Ropar
        \item Academics in terms of courses/ Instructors
        \item Re-alive the Wiki of IIT Ropar
    \end{itemize}
    \item Raise issues directly to corresponding Representatives of Mess, Sports, Hostel
\end{itemize}
\paragraph{Inspiration}
Inconvenience at institute due to only email being the only source of information. Leads to spamming and sometimes missing the relevant the information. We don't wish to replace email but integrate our app to help organise - Maybe replace ``Gentle Reminders".

\clearpage

\section{Decision Maker}
\paragraph{Objective}
Provides logical flow for taking any relevant decision based on
\begin{itemize}
    \item Past Experience of user - Weighted on closeness to current situation
    \item Suggestions from others to user- Weighted on closeness to that person (from user e.g family member, friend, relative)
    \item Future Goals of user
\end{itemize}
\paragraph{Features}
It analyses the model of taking decision close to human. The apps doesn't take decision but provide a flow for taking decision. It gives `utility' of both decisions. 
\paragraph{Inspiration}
We always resent ``The Road Not Taken" but what if we provide solution; the user can move forward.

\vspace{1cm}

\section{Doctor App}
\paragraph{Objective}
To predict and diagnose the disease based on symptoms, past data of similar patients
\paragraph{Features}
\begin{itemize}
    \item Predict ailment
    \item Alert when to seek a doctor and which kind of doctor
    \item Lists the past data of users in terms of percentage successfully recovering from it
    \item Suggest viable remedies
\end{itemize}
\paragraph{Inspiration}
Very few persons are able to consult a doctor of right type in right time.

\clearpage

\section{News Sharing App}
\paragraph{Objective}
To bring people with similar interests in news together, and promote mutual discussions on current affairs.
\paragraph{Target Audience}
Professors and their students, various online communities, educational institutions, any random groups etc.
\paragraph{Inspiration}
Sharing news with people having similar interests, or goals. \\
For example: If I and certain friends of mine are preparing for civil services and we want to share a certain news clipping among us all, then this app can come handy. 

\paragraph{Layout}
This app will have certain groups, having group editors as well as members.
Group editors will have to contribute to the newspaper in the form of news clippings taken from other dailies or small blogs.
Members can suggest certain news, but this has to verified by a certain editor. 

\vspace{1cm}

\section{Voice Separator}
\paragraph{Objective}
Helps to separate tracks of various frequency from one audio source
\paragraph{Target Audience}
Music Professionals as well as amateurs, karaoke singers, music technicians, etc.
\paragraph{Features}
\begin{itemize}
    \item Build karaoke tracks
    \item Separates the 2 voices in a call recording
    \item Background music from a conversation
\end{itemize}

\vspace{1cm}

  



\end{document}
